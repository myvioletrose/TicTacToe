
% Default to the notebook output style

    


% Inherit from the specified cell style.




    
\documentclass[11pt]{article}

    
    
    \usepackage[T1]{fontenc}
    % Nicer default font (+ math font) than Computer Modern for most use cases
    \usepackage{mathpazo}

    % Basic figure setup, for now with no caption control since it's done
    % automatically by Pandoc (which extracts ![](path) syntax from Markdown).
    \usepackage{graphicx}
    % We will generate all images so they have a width \maxwidth. This means
    % that they will get their normal width if they fit onto the page, but
    % are scaled down if they would overflow the margins.
    \makeatletter
    \def\maxwidth{\ifdim\Gin@nat@width>\linewidth\linewidth
    \else\Gin@nat@width\fi}
    \makeatother
    \let\Oldincludegraphics\includegraphics
    % Set max figure width to be 80% of text width, for now hardcoded.
    \renewcommand{\includegraphics}[1]{\Oldincludegraphics[width=.8\maxwidth]{#1}}
    % Ensure that by default, figures have no caption (until we provide a
    % proper Figure object with a Caption API and a way to capture that
    % in the conversion process - todo).
    \usepackage{caption}
    \DeclareCaptionLabelFormat{nolabel}{}
    \captionsetup{labelformat=nolabel}

    \usepackage{adjustbox} % Used to constrain images to a maximum size 
    \usepackage{xcolor} % Allow colors to be defined
    \usepackage{enumerate} % Needed for markdown enumerations to work
    \usepackage{geometry} % Used to adjust the document margins
    \usepackage{amsmath} % Equations
    \usepackage{amssymb} % Equations
    \usepackage{textcomp} % defines textquotesingle
    % Hack from http://tex.stackexchange.com/a/47451/13684:
    \AtBeginDocument{%
        \def\PYZsq{\textquotesingle}% Upright quotes in Pygmentized code
    }
    \usepackage{upquote} % Upright quotes for verbatim code
    \usepackage{eurosym} % defines \euro
    \usepackage[mathletters]{ucs} % Extended unicode (utf-8) support
    \usepackage[utf8x]{inputenc} % Allow utf-8 characters in the tex document
    \usepackage{fancyvrb} % verbatim replacement that allows latex
    \usepackage{grffile} % extends the file name processing of package graphics 
                         % to support a larger range 
    % The hyperref package gives us a pdf with properly built
    % internal navigation ('pdf bookmarks' for the table of contents,
    % internal cross-reference links, web links for URLs, etc.)
    \usepackage{hyperref}
    \usepackage{longtable} % longtable support required by pandoc >1.10
    \usepackage{booktabs}  % table support for pandoc > 1.12.2
    \usepackage[inline]{enumitem} % IRkernel/repr support (it uses the enumerate* environment)
    \usepackage[normalem]{ulem} % ulem is needed to support strikethroughs (\sout)
                                % normalem makes italics be italics, not underlines
    

    
    
    % Colors for the hyperref package
    \definecolor{urlcolor}{rgb}{0,.145,.698}
    \definecolor{linkcolor}{rgb}{.71,0.21,0.01}
    \definecolor{citecolor}{rgb}{.12,.54,.11}

    % ANSI colors
    \definecolor{ansi-black}{HTML}{3E424D}
    \definecolor{ansi-black-intense}{HTML}{282C36}
    \definecolor{ansi-red}{HTML}{E75C58}
    \definecolor{ansi-red-intense}{HTML}{B22B31}
    \definecolor{ansi-green}{HTML}{00A250}
    \definecolor{ansi-green-intense}{HTML}{007427}
    \definecolor{ansi-yellow}{HTML}{DDB62B}
    \definecolor{ansi-yellow-intense}{HTML}{B27D12}
    \definecolor{ansi-blue}{HTML}{208FFB}
    \definecolor{ansi-blue-intense}{HTML}{0065CA}
    \definecolor{ansi-magenta}{HTML}{D160C4}
    \definecolor{ansi-magenta-intense}{HTML}{A03196}
    \definecolor{ansi-cyan}{HTML}{60C6C8}
    \definecolor{ansi-cyan-intense}{HTML}{258F8F}
    \definecolor{ansi-white}{HTML}{C5C1B4}
    \definecolor{ansi-white-intense}{HTML}{A1A6B2}

    % commands and environments needed by pandoc snippets
    % extracted from the output of `pandoc -s`
    \providecommand{\tightlist}{%
      \setlength{\itemsep}{0pt}\setlength{\parskip}{0pt}}
    \DefineVerbatimEnvironment{Highlighting}{Verbatim}{commandchars=\\\{\}}
    % Add ',fontsize=\small' for more characters per line
    \newenvironment{Shaded}{}{}
    \newcommand{\KeywordTok}[1]{\textcolor[rgb]{0.00,0.44,0.13}{\textbf{{#1}}}}
    \newcommand{\DataTypeTok}[1]{\textcolor[rgb]{0.56,0.13,0.00}{{#1}}}
    \newcommand{\DecValTok}[1]{\textcolor[rgb]{0.25,0.63,0.44}{{#1}}}
    \newcommand{\BaseNTok}[1]{\textcolor[rgb]{0.25,0.63,0.44}{{#1}}}
    \newcommand{\FloatTok}[1]{\textcolor[rgb]{0.25,0.63,0.44}{{#1}}}
    \newcommand{\CharTok}[1]{\textcolor[rgb]{0.25,0.44,0.63}{{#1}}}
    \newcommand{\StringTok}[1]{\textcolor[rgb]{0.25,0.44,0.63}{{#1}}}
    \newcommand{\CommentTok}[1]{\textcolor[rgb]{0.38,0.63,0.69}{\textit{{#1}}}}
    \newcommand{\OtherTok}[1]{\textcolor[rgb]{0.00,0.44,0.13}{{#1}}}
    \newcommand{\AlertTok}[1]{\textcolor[rgb]{1.00,0.00,0.00}{\textbf{{#1}}}}
    \newcommand{\FunctionTok}[1]{\textcolor[rgb]{0.02,0.16,0.49}{{#1}}}
    \newcommand{\RegionMarkerTok}[1]{{#1}}
    \newcommand{\ErrorTok}[1]{\textcolor[rgb]{1.00,0.00,0.00}{\textbf{{#1}}}}
    \newcommand{\NormalTok}[1]{{#1}}
    
    % Additional commands for more recent versions of Pandoc
    \newcommand{\ConstantTok}[1]{\textcolor[rgb]{0.53,0.00,0.00}{{#1}}}
    \newcommand{\SpecialCharTok}[1]{\textcolor[rgb]{0.25,0.44,0.63}{{#1}}}
    \newcommand{\VerbatimStringTok}[1]{\textcolor[rgb]{0.25,0.44,0.63}{{#1}}}
    \newcommand{\SpecialStringTok}[1]{\textcolor[rgb]{0.73,0.40,0.53}{{#1}}}
    \newcommand{\ImportTok}[1]{{#1}}
    \newcommand{\DocumentationTok}[1]{\textcolor[rgb]{0.73,0.13,0.13}{\textit{{#1}}}}
    \newcommand{\AnnotationTok}[1]{\textcolor[rgb]{0.38,0.63,0.69}{\textbf{\textit{{#1}}}}}
    \newcommand{\CommentVarTok}[1]{\textcolor[rgb]{0.38,0.63,0.69}{\textbf{\textit{{#1}}}}}
    \newcommand{\VariableTok}[1]{\textcolor[rgb]{0.10,0.09,0.49}{{#1}}}
    \newcommand{\ControlFlowTok}[1]{\textcolor[rgb]{0.00,0.44,0.13}{\textbf{{#1}}}}
    \newcommand{\OperatorTok}[1]{\textcolor[rgb]{0.40,0.40,0.40}{{#1}}}
    \newcommand{\BuiltInTok}[1]{{#1}}
    \newcommand{\ExtensionTok}[1]{{#1}}
    \newcommand{\PreprocessorTok}[1]{\textcolor[rgb]{0.74,0.48,0.00}{{#1}}}
    \newcommand{\AttributeTok}[1]{\textcolor[rgb]{0.49,0.56,0.16}{{#1}}}
    \newcommand{\InformationTok}[1]{\textcolor[rgb]{0.38,0.63,0.69}{\textbf{\textit{{#1}}}}}
    \newcommand{\WarningTok}[1]{\textcolor[rgb]{0.38,0.63,0.69}{\textbf{\textit{{#1}}}}}
    
    
    % Define a nice break command that doesn't care if a line doesn't already
    % exist.
    \def\br{\hspace*{\fill} \\* }
    % Math Jax compatability definitions
    \def\gt{>}
    \def\lt{<}
    % Document parameters
    \title{03-Milestone Project 1 - Complete Walkthrough Solution}
    
    
    

    % Pygments definitions
    
\makeatletter
\def\PY@reset{\let\PY@it=\relax \let\PY@bf=\relax%
    \let\PY@ul=\relax \let\PY@tc=\relax%
    \let\PY@bc=\relax \let\PY@ff=\relax}
\def\PY@tok#1{\csname PY@tok@#1\endcsname}
\def\PY@toks#1+{\ifx\relax#1\empty\else%
    \PY@tok{#1}\expandafter\PY@toks\fi}
\def\PY@do#1{\PY@bc{\PY@tc{\PY@ul{%
    \PY@it{\PY@bf{\PY@ff{#1}}}}}}}
\def\PY#1#2{\PY@reset\PY@toks#1+\relax+\PY@do{#2}}

\expandafter\def\csname PY@tok@w\endcsname{\def\PY@tc##1{\textcolor[rgb]{0.73,0.73,0.73}{##1}}}
\expandafter\def\csname PY@tok@c\endcsname{\let\PY@it=\textit\def\PY@tc##1{\textcolor[rgb]{0.25,0.50,0.50}{##1}}}
\expandafter\def\csname PY@tok@cp\endcsname{\def\PY@tc##1{\textcolor[rgb]{0.74,0.48,0.00}{##1}}}
\expandafter\def\csname PY@tok@k\endcsname{\let\PY@bf=\textbf\def\PY@tc##1{\textcolor[rgb]{0.00,0.50,0.00}{##1}}}
\expandafter\def\csname PY@tok@kp\endcsname{\def\PY@tc##1{\textcolor[rgb]{0.00,0.50,0.00}{##1}}}
\expandafter\def\csname PY@tok@kt\endcsname{\def\PY@tc##1{\textcolor[rgb]{0.69,0.00,0.25}{##1}}}
\expandafter\def\csname PY@tok@o\endcsname{\def\PY@tc##1{\textcolor[rgb]{0.40,0.40,0.40}{##1}}}
\expandafter\def\csname PY@tok@ow\endcsname{\let\PY@bf=\textbf\def\PY@tc##1{\textcolor[rgb]{0.67,0.13,1.00}{##1}}}
\expandafter\def\csname PY@tok@nb\endcsname{\def\PY@tc##1{\textcolor[rgb]{0.00,0.50,0.00}{##1}}}
\expandafter\def\csname PY@tok@nf\endcsname{\def\PY@tc##1{\textcolor[rgb]{0.00,0.00,1.00}{##1}}}
\expandafter\def\csname PY@tok@nc\endcsname{\let\PY@bf=\textbf\def\PY@tc##1{\textcolor[rgb]{0.00,0.00,1.00}{##1}}}
\expandafter\def\csname PY@tok@nn\endcsname{\let\PY@bf=\textbf\def\PY@tc##1{\textcolor[rgb]{0.00,0.00,1.00}{##1}}}
\expandafter\def\csname PY@tok@ne\endcsname{\let\PY@bf=\textbf\def\PY@tc##1{\textcolor[rgb]{0.82,0.25,0.23}{##1}}}
\expandafter\def\csname PY@tok@nv\endcsname{\def\PY@tc##1{\textcolor[rgb]{0.10,0.09,0.49}{##1}}}
\expandafter\def\csname PY@tok@no\endcsname{\def\PY@tc##1{\textcolor[rgb]{0.53,0.00,0.00}{##1}}}
\expandafter\def\csname PY@tok@nl\endcsname{\def\PY@tc##1{\textcolor[rgb]{0.63,0.63,0.00}{##1}}}
\expandafter\def\csname PY@tok@ni\endcsname{\let\PY@bf=\textbf\def\PY@tc##1{\textcolor[rgb]{0.60,0.60,0.60}{##1}}}
\expandafter\def\csname PY@tok@na\endcsname{\def\PY@tc##1{\textcolor[rgb]{0.49,0.56,0.16}{##1}}}
\expandafter\def\csname PY@tok@nt\endcsname{\let\PY@bf=\textbf\def\PY@tc##1{\textcolor[rgb]{0.00,0.50,0.00}{##1}}}
\expandafter\def\csname PY@tok@nd\endcsname{\def\PY@tc##1{\textcolor[rgb]{0.67,0.13,1.00}{##1}}}
\expandafter\def\csname PY@tok@s\endcsname{\def\PY@tc##1{\textcolor[rgb]{0.73,0.13,0.13}{##1}}}
\expandafter\def\csname PY@tok@sd\endcsname{\let\PY@it=\textit\def\PY@tc##1{\textcolor[rgb]{0.73,0.13,0.13}{##1}}}
\expandafter\def\csname PY@tok@si\endcsname{\let\PY@bf=\textbf\def\PY@tc##1{\textcolor[rgb]{0.73,0.40,0.53}{##1}}}
\expandafter\def\csname PY@tok@se\endcsname{\let\PY@bf=\textbf\def\PY@tc##1{\textcolor[rgb]{0.73,0.40,0.13}{##1}}}
\expandafter\def\csname PY@tok@sr\endcsname{\def\PY@tc##1{\textcolor[rgb]{0.73,0.40,0.53}{##1}}}
\expandafter\def\csname PY@tok@ss\endcsname{\def\PY@tc##1{\textcolor[rgb]{0.10,0.09,0.49}{##1}}}
\expandafter\def\csname PY@tok@sx\endcsname{\def\PY@tc##1{\textcolor[rgb]{0.00,0.50,0.00}{##1}}}
\expandafter\def\csname PY@tok@m\endcsname{\def\PY@tc##1{\textcolor[rgb]{0.40,0.40,0.40}{##1}}}
\expandafter\def\csname PY@tok@gh\endcsname{\let\PY@bf=\textbf\def\PY@tc##1{\textcolor[rgb]{0.00,0.00,0.50}{##1}}}
\expandafter\def\csname PY@tok@gu\endcsname{\let\PY@bf=\textbf\def\PY@tc##1{\textcolor[rgb]{0.50,0.00,0.50}{##1}}}
\expandafter\def\csname PY@tok@gd\endcsname{\def\PY@tc##1{\textcolor[rgb]{0.63,0.00,0.00}{##1}}}
\expandafter\def\csname PY@tok@gi\endcsname{\def\PY@tc##1{\textcolor[rgb]{0.00,0.63,0.00}{##1}}}
\expandafter\def\csname PY@tok@gr\endcsname{\def\PY@tc##1{\textcolor[rgb]{1.00,0.00,0.00}{##1}}}
\expandafter\def\csname PY@tok@ge\endcsname{\let\PY@it=\textit}
\expandafter\def\csname PY@tok@gs\endcsname{\let\PY@bf=\textbf}
\expandafter\def\csname PY@tok@gp\endcsname{\let\PY@bf=\textbf\def\PY@tc##1{\textcolor[rgb]{0.00,0.00,0.50}{##1}}}
\expandafter\def\csname PY@tok@go\endcsname{\def\PY@tc##1{\textcolor[rgb]{0.53,0.53,0.53}{##1}}}
\expandafter\def\csname PY@tok@gt\endcsname{\def\PY@tc##1{\textcolor[rgb]{0.00,0.27,0.87}{##1}}}
\expandafter\def\csname PY@tok@err\endcsname{\def\PY@bc##1{\setlength{\fboxsep}{0pt}\fcolorbox[rgb]{1.00,0.00,0.00}{1,1,1}{\strut ##1}}}
\expandafter\def\csname PY@tok@kc\endcsname{\let\PY@bf=\textbf\def\PY@tc##1{\textcolor[rgb]{0.00,0.50,0.00}{##1}}}
\expandafter\def\csname PY@tok@kd\endcsname{\let\PY@bf=\textbf\def\PY@tc##1{\textcolor[rgb]{0.00,0.50,0.00}{##1}}}
\expandafter\def\csname PY@tok@kn\endcsname{\let\PY@bf=\textbf\def\PY@tc##1{\textcolor[rgb]{0.00,0.50,0.00}{##1}}}
\expandafter\def\csname PY@tok@kr\endcsname{\let\PY@bf=\textbf\def\PY@tc##1{\textcolor[rgb]{0.00,0.50,0.00}{##1}}}
\expandafter\def\csname PY@tok@bp\endcsname{\def\PY@tc##1{\textcolor[rgb]{0.00,0.50,0.00}{##1}}}
\expandafter\def\csname PY@tok@fm\endcsname{\def\PY@tc##1{\textcolor[rgb]{0.00,0.00,1.00}{##1}}}
\expandafter\def\csname PY@tok@vc\endcsname{\def\PY@tc##1{\textcolor[rgb]{0.10,0.09,0.49}{##1}}}
\expandafter\def\csname PY@tok@vg\endcsname{\def\PY@tc##1{\textcolor[rgb]{0.10,0.09,0.49}{##1}}}
\expandafter\def\csname PY@tok@vi\endcsname{\def\PY@tc##1{\textcolor[rgb]{0.10,0.09,0.49}{##1}}}
\expandafter\def\csname PY@tok@vm\endcsname{\def\PY@tc##1{\textcolor[rgb]{0.10,0.09,0.49}{##1}}}
\expandafter\def\csname PY@tok@sa\endcsname{\def\PY@tc##1{\textcolor[rgb]{0.73,0.13,0.13}{##1}}}
\expandafter\def\csname PY@tok@sb\endcsname{\def\PY@tc##1{\textcolor[rgb]{0.73,0.13,0.13}{##1}}}
\expandafter\def\csname PY@tok@sc\endcsname{\def\PY@tc##1{\textcolor[rgb]{0.73,0.13,0.13}{##1}}}
\expandafter\def\csname PY@tok@dl\endcsname{\def\PY@tc##1{\textcolor[rgb]{0.73,0.13,0.13}{##1}}}
\expandafter\def\csname PY@tok@s2\endcsname{\def\PY@tc##1{\textcolor[rgb]{0.73,0.13,0.13}{##1}}}
\expandafter\def\csname PY@tok@sh\endcsname{\def\PY@tc##1{\textcolor[rgb]{0.73,0.13,0.13}{##1}}}
\expandafter\def\csname PY@tok@s1\endcsname{\def\PY@tc##1{\textcolor[rgb]{0.73,0.13,0.13}{##1}}}
\expandafter\def\csname PY@tok@mb\endcsname{\def\PY@tc##1{\textcolor[rgb]{0.40,0.40,0.40}{##1}}}
\expandafter\def\csname PY@tok@mf\endcsname{\def\PY@tc##1{\textcolor[rgb]{0.40,0.40,0.40}{##1}}}
\expandafter\def\csname PY@tok@mh\endcsname{\def\PY@tc##1{\textcolor[rgb]{0.40,0.40,0.40}{##1}}}
\expandafter\def\csname PY@tok@mi\endcsname{\def\PY@tc##1{\textcolor[rgb]{0.40,0.40,0.40}{##1}}}
\expandafter\def\csname PY@tok@il\endcsname{\def\PY@tc##1{\textcolor[rgb]{0.40,0.40,0.40}{##1}}}
\expandafter\def\csname PY@tok@mo\endcsname{\def\PY@tc##1{\textcolor[rgb]{0.40,0.40,0.40}{##1}}}
\expandafter\def\csname PY@tok@ch\endcsname{\let\PY@it=\textit\def\PY@tc##1{\textcolor[rgb]{0.25,0.50,0.50}{##1}}}
\expandafter\def\csname PY@tok@cm\endcsname{\let\PY@it=\textit\def\PY@tc##1{\textcolor[rgb]{0.25,0.50,0.50}{##1}}}
\expandafter\def\csname PY@tok@cpf\endcsname{\let\PY@it=\textit\def\PY@tc##1{\textcolor[rgb]{0.25,0.50,0.50}{##1}}}
\expandafter\def\csname PY@tok@c1\endcsname{\let\PY@it=\textit\def\PY@tc##1{\textcolor[rgb]{0.25,0.50,0.50}{##1}}}
\expandafter\def\csname PY@tok@cs\endcsname{\let\PY@it=\textit\def\PY@tc##1{\textcolor[rgb]{0.25,0.50,0.50}{##1}}}

\def\PYZbs{\char`\\}
\def\PYZus{\char`\_}
\def\PYZob{\char`\{}
\def\PYZcb{\char`\}}
\def\PYZca{\char`\^}
\def\PYZam{\char`\&}
\def\PYZlt{\char`\<}
\def\PYZgt{\char`\>}
\def\PYZsh{\char`\#}
\def\PYZpc{\char`\%}
\def\PYZdl{\char`\$}
\def\PYZhy{\char`\-}
\def\PYZsq{\char`\'}
\def\PYZdq{\char`\"}
\def\PYZti{\char`\~}
% for compatibility with earlier versions
\def\PYZat{@}
\def\PYZlb{[}
\def\PYZrb{]}
\makeatother


    % Exact colors from NB
    \definecolor{incolor}{rgb}{0.0, 0.0, 0.5}
    \definecolor{outcolor}{rgb}{0.545, 0.0, 0.0}



    
    % Prevent overflowing lines due to hard-to-break entities
    \sloppy 
    % Setup hyperref package
    \hypersetup{
      breaklinks=true,  % so long urls are correctly broken across lines
      colorlinks=true,
      urlcolor=urlcolor,
      linkcolor=linkcolor,
      citecolor=citecolor,
      }
    % Slightly bigger margins than the latex defaults
    
    \geometry{verbose,tmargin=1in,bmargin=1in,lmargin=1in,rmargin=1in}
    
    

    \begin{document}
    
    
    \maketitle
    
    

    
    \section{Milestone Project 1: Full Walk-through Code
Solution}\label{milestone-project-1-full-walk-through-code-solution}

Below is the filled in code that goes along with the complete
walk-through video. Check out the corresponding lecture videos for more
information on this code!

    \textbf{Step 1: Write a function that can print out a board. Set up your
board as a list, where each index 1-9 corresponds with a number on a
number pad, so you get a 3 by 3 board representation.}

    \begin{Verbatim}[commandchars=\\\{\}]
{\color{incolor}In [{\color{incolor}1}]:} \PY{k+kn}{from} \PY{n+nn}{IPython}\PY{n+nn}{.}\PY{n+nn}{display} \PY{k}{import} \PY{n}{clear\PYZus{}output}
        
        \PY{k}{def} \PY{n+nf}{display\PYZus{}board}\PY{p}{(}\PY{n}{board}\PY{p}{)}\PY{p}{:}
            \PY{n}{clear\PYZus{}output}\PY{p}{(}\PY{p}{)}  \PY{c+c1}{\PYZsh{} Remember, this only works in jupyter!}
            
            \PY{n+nb}{print}\PY{p}{(}\PY{l+s+s1}{\PYZsq{}}\PY{l+s+s1}{   |   |}\PY{l+s+s1}{\PYZsq{}}\PY{p}{)}
            \PY{n+nb}{print}\PY{p}{(}\PY{l+s+s1}{\PYZsq{}}\PY{l+s+s1}{ }\PY{l+s+s1}{\PYZsq{}} \PY{o}{+} \PY{n}{board}\PY{p}{[}\PY{l+m+mi}{7}\PY{p}{]} \PY{o}{+} \PY{l+s+s1}{\PYZsq{}}\PY{l+s+s1}{ | }\PY{l+s+s1}{\PYZsq{}} \PY{o}{+} \PY{n}{board}\PY{p}{[}\PY{l+m+mi}{8}\PY{p}{]} \PY{o}{+} \PY{l+s+s1}{\PYZsq{}}\PY{l+s+s1}{ | }\PY{l+s+s1}{\PYZsq{}} \PY{o}{+} \PY{n}{board}\PY{p}{[}\PY{l+m+mi}{9}\PY{p}{]}\PY{p}{)}
            \PY{n+nb}{print}\PY{p}{(}\PY{l+s+s1}{\PYZsq{}}\PY{l+s+s1}{   |   |}\PY{l+s+s1}{\PYZsq{}}\PY{p}{)}
            \PY{n+nb}{print}\PY{p}{(}\PY{l+s+s1}{\PYZsq{}}\PY{l+s+s1}{\PYZhy{}\PYZhy{}\PYZhy{}\PYZhy{}\PYZhy{}\PYZhy{}\PYZhy{}\PYZhy{}\PYZhy{}\PYZhy{}\PYZhy{}}\PY{l+s+s1}{\PYZsq{}}\PY{p}{)}
            \PY{n+nb}{print}\PY{p}{(}\PY{l+s+s1}{\PYZsq{}}\PY{l+s+s1}{   |   |}\PY{l+s+s1}{\PYZsq{}}\PY{p}{)}
            \PY{n+nb}{print}\PY{p}{(}\PY{l+s+s1}{\PYZsq{}}\PY{l+s+s1}{ }\PY{l+s+s1}{\PYZsq{}} \PY{o}{+} \PY{n}{board}\PY{p}{[}\PY{l+m+mi}{4}\PY{p}{]} \PY{o}{+} \PY{l+s+s1}{\PYZsq{}}\PY{l+s+s1}{ | }\PY{l+s+s1}{\PYZsq{}} \PY{o}{+} \PY{n}{board}\PY{p}{[}\PY{l+m+mi}{5}\PY{p}{]} \PY{o}{+} \PY{l+s+s1}{\PYZsq{}}\PY{l+s+s1}{ | }\PY{l+s+s1}{\PYZsq{}} \PY{o}{+} \PY{n}{board}\PY{p}{[}\PY{l+m+mi}{6}\PY{p}{]}\PY{p}{)}
            \PY{n+nb}{print}\PY{p}{(}\PY{l+s+s1}{\PYZsq{}}\PY{l+s+s1}{   |   |}\PY{l+s+s1}{\PYZsq{}}\PY{p}{)}
            \PY{n+nb}{print}\PY{p}{(}\PY{l+s+s1}{\PYZsq{}}\PY{l+s+s1}{\PYZhy{}\PYZhy{}\PYZhy{}\PYZhy{}\PYZhy{}\PYZhy{}\PYZhy{}\PYZhy{}\PYZhy{}\PYZhy{}\PYZhy{}}\PY{l+s+s1}{\PYZsq{}}\PY{p}{)}
            \PY{n+nb}{print}\PY{p}{(}\PY{l+s+s1}{\PYZsq{}}\PY{l+s+s1}{   |   |}\PY{l+s+s1}{\PYZsq{}}\PY{p}{)}
            \PY{n+nb}{print}\PY{p}{(}\PY{l+s+s1}{\PYZsq{}}\PY{l+s+s1}{ }\PY{l+s+s1}{\PYZsq{}} \PY{o}{+} \PY{n}{board}\PY{p}{[}\PY{l+m+mi}{1}\PY{p}{]} \PY{o}{+} \PY{l+s+s1}{\PYZsq{}}\PY{l+s+s1}{ | }\PY{l+s+s1}{\PYZsq{}} \PY{o}{+} \PY{n}{board}\PY{p}{[}\PY{l+m+mi}{2}\PY{p}{]} \PY{o}{+} \PY{l+s+s1}{\PYZsq{}}\PY{l+s+s1}{ | }\PY{l+s+s1}{\PYZsq{}} \PY{o}{+} \PY{n}{board}\PY{p}{[}\PY{l+m+mi}{3}\PY{p}{]}\PY{p}{)}
            \PY{n+nb}{print}\PY{p}{(}\PY{l+s+s1}{\PYZsq{}}\PY{l+s+s1}{   |   |}\PY{l+s+s1}{\PYZsq{}}\PY{p}{)}
\end{Verbatim}


    \textbf{TEST Step 1:} run your function on a test version of the board
list, and make adjustments as necessary

    \begin{Verbatim}[commandchars=\\\{\}]
{\color{incolor}In [{\color{incolor}2}]:} \PY{n}{test\PYZus{}board} \PY{o}{=} \PY{p}{[}\PY{l+s+s1}{\PYZsq{}}\PY{l+s+s1}{\PYZsh{}}\PY{l+s+s1}{\PYZsq{}}\PY{p}{,}\PY{l+s+s1}{\PYZsq{}}\PY{l+s+s1}{X}\PY{l+s+s1}{\PYZsq{}}\PY{p}{,}\PY{l+s+s1}{\PYZsq{}}\PY{l+s+s1}{O}\PY{l+s+s1}{\PYZsq{}}\PY{p}{,}\PY{l+s+s1}{\PYZsq{}}\PY{l+s+s1}{X}\PY{l+s+s1}{\PYZsq{}}\PY{p}{,}\PY{l+s+s1}{\PYZsq{}}\PY{l+s+s1}{O}\PY{l+s+s1}{\PYZsq{}}\PY{p}{,}\PY{l+s+s1}{\PYZsq{}}\PY{l+s+s1}{X}\PY{l+s+s1}{\PYZsq{}}\PY{p}{,}\PY{l+s+s1}{\PYZsq{}}\PY{l+s+s1}{O}\PY{l+s+s1}{\PYZsq{}}\PY{p}{,}\PY{l+s+s1}{\PYZsq{}}\PY{l+s+s1}{X}\PY{l+s+s1}{\PYZsq{}}\PY{p}{,}\PY{l+s+s1}{\PYZsq{}}\PY{l+s+s1}{O}\PY{l+s+s1}{\PYZsq{}}\PY{p}{,}\PY{l+s+s1}{\PYZsq{}}\PY{l+s+s1}{X}\PY{l+s+s1}{\PYZsq{}}\PY{p}{]}
        \PY{n}{display\PYZus{}board}\PY{p}{(}\PY{n}{test\PYZus{}board}\PY{p}{)}
\end{Verbatim}


    \begin{Verbatim}[commandchars=\\\{\}]
   |   |
 X | O | X
   |   |
-----------
   |   |
 O | X | O
   |   |
-----------
   |   |
 X | O | X
   |   |

    \end{Verbatim}

    \textbf{Step 2: Write a function that can take in a player input and
assign their marker as 'X' or 'O'. Think about using \emph{while} loops
to continually ask until you get a correct answer.}

    \begin{Verbatim}[commandchars=\\\{\}]
{\color{incolor}In [{\color{incolor}3}]:} \PY{k}{def} \PY{n+nf}{player\PYZus{}input}\PY{p}{(}\PY{p}{)}\PY{p}{:}
            \PY{n}{marker} \PY{o}{=} \PY{l+s+s1}{\PYZsq{}}\PY{l+s+s1}{\PYZsq{}}
            
            \PY{k}{while} \PY{o+ow}{not} \PY{p}{(}\PY{n}{marker} \PY{o}{==} \PY{l+s+s1}{\PYZsq{}}\PY{l+s+s1}{X}\PY{l+s+s1}{\PYZsq{}} \PY{o+ow}{or} \PY{n}{marker} \PY{o}{==} \PY{l+s+s1}{\PYZsq{}}\PY{l+s+s1}{O}\PY{l+s+s1}{\PYZsq{}}\PY{p}{)}\PY{p}{:}
                \PY{n}{marker} \PY{o}{=} \PY{n+nb}{input}\PY{p}{(}\PY{l+s+s1}{\PYZsq{}}\PY{l+s+s1}{Player 1: Do you want to be X or O? }\PY{l+s+s1}{\PYZsq{}}\PY{p}{)}\PY{o}{.}\PY{n}{upper}\PY{p}{(}\PY{p}{)}
        
            \PY{k}{if} \PY{n}{marker} \PY{o}{==} \PY{l+s+s1}{\PYZsq{}}\PY{l+s+s1}{X}\PY{l+s+s1}{\PYZsq{}}\PY{p}{:}
                \PY{k}{return} \PY{p}{(}\PY{l+s+s1}{\PYZsq{}}\PY{l+s+s1}{X}\PY{l+s+s1}{\PYZsq{}}\PY{p}{,} \PY{l+s+s1}{\PYZsq{}}\PY{l+s+s1}{O}\PY{l+s+s1}{\PYZsq{}}\PY{p}{)}
            \PY{k}{else}\PY{p}{:}
                \PY{k}{return} \PY{p}{(}\PY{l+s+s1}{\PYZsq{}}\PY{l+s+s1}{O}\PY{l+s+s1}{\PYZsq{}}\PY{p}{,} \PY{l+s+s1}{\PYZsq{}}\PY{l+s+s1}{X}\PY{l+s+s1}{\PYZsq{}}\PY{p}{)}
\end{Verbatim}


    \textbf{TEST Step 2:} run the function to make sure it returns the
desired output

    \begin{Verbatim}[commandchars=\\\{\}]
{\color{incolor}In [{\color{incolor}4}]:} \PY{n}{player\PYZus{}input}\PY{p}{(}\PY{p}{)}
\end{Verbatim}


    \begin{Verbatim}[commandchars=\\\{\}]
Player 1: Do you want to be X or O? X

    \end{Verbatim}

\begin{Verbatim}[commandchars=\\\{\}]
{\color{outcolor}Out[{\color{outcolor}4}]:} ('X', 'O')
\end{Verbatim}
            
    \textbf{Step 3: Write a function that takes in the board list object, a
marker ('X' or 'O'), and a desired position (number 1-9) and assigns it
to the board.}

    \begin{Verbatim}[commandchars=\\\{\}]
{\color{incolor}In [{\color{incolor}5}]:} \PY{k}{def} \PY{n+nf}{place\PYZus{}marker}\PY{p}{(}\PY{n}{board}\PY{p}{,} \PY{n}{marker}\PY{p}{,} \PY{n}{position}\PY{p}{)}\PY{p}{:}
            \PY{n}{board}\PY{p}{[}\PY{n}{position}\PY{p}{]} \PY{o}{=} \PY{n}{marker}
\end{Verbatim}


    \textbf{TEST Step 3:} run the place marker function using test
parameters and display the modified board

    \begin{Verbatim}[commandchars=\\\{\}]
{\color{incolor}In [{\color{incolor}6}]:} \PY{n}{place\PYZus{}marker}\PY{p}{(}\PY{n}{test\PYZus{}board}\PY{p}{,}\PY{l+s+s1}{\PYZsq{}}\PY{l+s+s1}{\PYZdl{}}\PY{l+s+s1}{\PYZsq{}}\PY{p}{,}\PY{l+m+mi}{8}\PY{p}{)}
        \PY{n}{display\PYZus{}board}\PY{p}{(}\PY{n}{test\PYZus{}board}\PY{p}{)}
\end{Verbatim}


    \begin{Verbatim}[commandchars=\\\{\}]
   |   |
 X | \$ | X
   |   |
-----------
   |   |
 O | X | O
   |   |
-----------
   |   |
 X | O | X
   |   |

    \end{Verbatim}

    \textbf{Step 4: Write a function that takes in a board and checks to see
if someone has won. }

    \begin{Verbatim}[commandchars=\\\{\}]
{\color{incolor}In [{\color{incolor}7}]:} \PY{k}{def} \PY{n+nf}{win\PYZus{}check}\PY{p}{(}\PY{n}{board}\PY{p}{,}\PY{n}{mark}\PY{p}{)}\PY{p}{:}
            
            \PY{k}{return} \PY{p}{(}\PY{p}{(}\PY{n}{board}\PY{p}{[}\PY{l+m+mi}{7}\PY{p}{]} \PY{o}{==} \PY{n}{mark} \PY{o+ow}{and} \PY{n}{board}\PY{p}{[}\PY{l+m+mi}{8}\PY{p}{]} \PY{o}{==} \PY{n}{mark} \PY{o+ow}{and} \PY{n}{board}\PY{p}{[}\PY{l+m+mi}{9}\PY{p}{]} \PY{o}{==} \PY{n}{mark}\PY{p}{)} \PY{o+ow}{or} \PY{c+c1}{\PYZsh{} across the top}
            \PY{p}{(}\PY{n}{board}\PY{p}{[}\PY{l+m+mi}{4}\PY{p}{]} \PY{o}{==} \PY{n}{mark} \PY{o+ow}{and} \PY{n}{board}\PY{p}{[}\PY{l+m+mi}{5}\PY{p}{]} \PY{o}{==} \PY{n}{mark} \PY{o+ow}{and} \PY{n}{board}\PY{p}{[}\PY{l+m+mi}{6}\PY{p}{]} \PY{o}{==} \PY{n}{mark}\PY{p}{)} \PY{o+ow}{or} \PY{c+c1}{\PYZsh{} across the middle}
            \PY{p}{(}\PY{n}{board}\PY{p}{[}\PY{l+m+mi}{1}\PY{p}{]} \PY{o}{==} \PY{n}{mark} \PY{o+ow}{and} \PY{n}{board}\PY{p}{[}\PY{l+m+mi}{2}\PY{p}{]} \PY{o}{==} \PY{n}{mark} \PY{o+ow}{and} \PY{n}{board}\PY{p}{[}\PY{l+m+mi}{3}\PY{p}{]} \PY{o}{==} \PY{n}{mark}\PY{p}{)} \PY{o+ow}{or} \PY{c+c1}{\PYZsh{} across the bottom}
            \PY{p}{(}\PY{n}{board}\PY{p}{[}\PY{l+m+mi}{7}\PY{p}{]} \PY{o}{==} \PY{n}{mark} \PY{o+ow}{and} \PY{n}{board}\PY{p}{[}\PY{l+m+mi}{4}\PY{p}{]} \PY{o}{==} \PY{n}{mark} \PY{o+ow}{and} \PY{n}{board}\PY{p}{[}\PY{l+m+mi}{1}\PY{p}{]} \PY{o}{==} \PY{n}{mark}\PY{p}{)} \PY{o+ow}{or} \PY{c+c1}{\PYZsh{} down the middle}
            \PY{p}{(}\PY{n}{board}\PY{p}{[}\PY{l+m+mi}{8}\PY{p}{]} \PY{o}{==} \PY{n}{mark} \PY{o+ow}{and} \PY{n}{board}\PY{p}{[}\PY{l+m+mi}{5}\PY{p}{]} \PY{o}{==} \PY{n}{mark} \PY{o+ow}{and} \PY{n}{board}\PY{p}{[}\PY{l+m+mi}{2}\PY{p}{]} \PY{o}{==} \PY{n}{mark}\PY{p}{)} \PY{o+ow}{or} \PY{c+c1}{\PYZsh{} down the middle}
            \PY{p}{(}\PY{n}{board}\PY{p}{[}\PY{l+m+mi}{9}\PY{p}{]} \PY{o}{==} \PY{n}{mark} \PY{o+ow}{and} \PY{n}{board}\PY{p}{[}\PY{l+m+mi}{6}\PY{p}{]} \PY{o}{==} \PY{n}{mark} \PY{o+ow}{and} \PY{n}{board}\PY{p}{[}\PY{l+m+mi}{3}\PY{p}{]} \PY{o}{==} \PY{n}{mark}\PY{p}{)} \PY{o+ow}{or} \PY{c+c1}{\PYZsh{} down the right side}
            \PY{p}{(}\PY{n}{board}\PY{p}{[}\PY{l+m+mi}{7}\PY{p}{]} \PY{o}{==} \PY{n}{mark} \PY{o+ow}{and} \PY{n}{board}\PY{p}{[}\PY{l+m+mi}{5}\PY{p}{]} \PY{o}{==} \PY{n}{mark} \PY{o+ow}{and} \PY{n}{board}\PY{p}{[}\PY{l+m+mi}{3}\PY{p}{]} \PY{o}{==} \PY{n}{mark}\PY{p}{)} \PY{o+ow}{or} \PY{c+c1}{\PYZsh{} diagonal}
            \PY{p}{(}\PY{n}{board}\PY{p}{[}\PY{l+m+mi}{9}\PY{p}{]} \PY{o}{==} \PY{n}{mark} \PY{o+ow}{and} \PY{n}{board}\PY{p}{[}\PY{l+m+mi}{5}\PY{p}{]} \PY{o}{==} \PY{n}{mark} \PY{o+ow}{and} \PY{n}{board}\PY{p}{[}\PY{l+m+mi}{1}\PY{p}{]} \PY{o}{==} \PY{n}{mark}\PY{p}{)}\PY{p}{)} \PY{c+c1}{\PYZsh{} diagonal}
\end{Verbatim}


    \textbf{TEST Step 4:} run the win\_check function against our
test\_board - it should return True

    \begin{Verbatim}[commandchars=\\\{\}]
{\color{incolor}In [{\color{incolor}8}]:} \PY{n}{win\PYZus{}check}\PY{p}{(}\PY{n}{test\PYZus{}board}\PY{p}{,}\PY{l+s+s1}{\PYZsq{}}\PY{l+s+s1}{X}\PY{l+s+s1}{\PYZsq{}}\PY{p}{)}
\end{Verbatim}


\begin{Verbatim}[commandchars=\\\{\}]
{\color{outcolor}Out[{\color{outcolor}8}]:} True
\end{Verbatim}
            
    \textbf{Step 5: Write a function that uses the random module to randomly
decide which player goes first. You may want to lookup random.randint()
Return a string of which player went first.}

    \begin{Verbatim}[commandchars=\\\{\}]
{\color{incolor}In [{\color{incolor}9}]:} \PY{k+kn}{import} \PY{n+nn}{random}
        
        \PY{k}{def} \PY{n+nf}{choose\PYZus{}first}\PY{p}{(}\PY{p}{)}\PY{p}{:}
            \PY{k}{if} \PY{n}{random}\PY{o}{.}\PY{n}{randint}\PY{p}{(}\PY{l+m+mi}{0}\PY{p}{,} \PY{l+m+mi}{1}\PY{p}{)} \PY{o}{==} \PY{l+m+mi}{0}\PY{p}{:}
                \PY{k}{return} \PY{l+s+s1}{\PYZsq{}}\PY{l+s+s1}{Player 2}\PY{l+s+s1}{\PYZsq{}}
            \PY{k}{else}\PY{p}{:}
                \PY{k}{return} \PY{l+s+s1}{\PYZsq{}}\PY{l+s+s1}{Player 1}\PY{l+s+s1}{\PYZsq{}}
\end{Verbatim}


    \textbf{Step 6: Write a function that returns a boolean indicating
whether a space on the board is freely available.}

    \begin{Verbatim}[commandchars=\\\{\}]
{\color{incolor}In [{\color{incolor}10}]:} \PY{k}{def} \PY{n+nf}{space\PYZus{}check}\PY{p}{(}\PY{n}{board}\PY{p}{,} \PY{n}{position}\PY{p}{)}\PY{p}{:}
             
             \PY{k}{return} \PY{n}{board}\PY{p}{[}\PY{n}{position}\PY{p}{]} \PY{o}{==} \PY{l+s+s1}{\PYZsq{}}\PY{l+s+s1}{ }\PY{l+s+s1}{\PYZsq{}}
\end{Verbatim}


    \textbf{Step 7: Write a function that checks if the board is full and
returns a boolean value. True if full, False otherwise.}

    \begin{Verbatim}[commandchars=\\\{\}]
{\color{incolor}In [{\color{incolor}11}]:} \PY{k}{def} \PY{n+nf}{full\PYZus{}board\PYZus{}check}\PY{p}{(}\PY{n}{board}\PY{p}{)}\PY{p}{:}
             \PY{k}{for} \PY{n}{i} \PY{o+ow}{in} \PY{n+nb}{range}\PY{p}{(}\PY{l+m+mi}{1}\PY{p}{,}\PY{l+m+mi}{10}\PY{p}{)}\PY{p}{:}
                 \PY{k}{if} \PY{n}{space\PYZus{}check}\PY{p}{(}\PY{n}{board}\PY{p}{,} \PY{n}{i}\PY{p}{)}\PY{p}{:}
                     \PY{k}{return} \PY{k+kc}{False}
             \PY{k}{return} \PY{k+kc}{True}
\end{Verbatim}


    \textbf{Step 8: Write a function that asks for a player's next position
(as a number 1-9) and then uses the function from step 6 to check if its
a free position. If it is, then return the position for later use. }

    \begin{Verbatim}[commandchars=\\\{\}]
{\color{incolor}In [{\color{incolor}12}]:} \PY{k}{def} \PY{n+nf}{player\PYZus{}choice}\PY{p}{(}\PY{n}{board}\PY{p}{)}\PY{p}{:}
             \PY{n}{position} \PY{o}{=} \PY{l+m+mi}{0}
             
             \PY{k}{while} \PY{n}{position} \PY{o+ow}{not} \PY{o+ow}{in} \PY{p}{[}\PY{l+m+mi}{1}\PY{p}{,}\PY{l+m+mi}{2}\PY{p}{,}\PY{l+m+mi}{3}\PY{p}{,}\PY{l+m+mi}{4}\PY{p}{,}\PY{l+m+mi}{5}\PY{p}{,}\PY{l+m+mi}{6}\PY{p}{,}\PY{l+m+mi}{7}\PY{p}{,}\PY{l+m+mi}{8}\PY{p}{,}\PY{l+m+mi}{9}\PY{p}{]} \PY{o+ow}{or} \PY{o+ow}{not} \PY{n}{space\PYZus{}check}\PY{p}{(}\PY{n}{board}\PY{p}{,} \PY{n}{position}\PY{p}{)}\PY{p}{:}
                 \PY{n}{position} \PY{o}{=} \PY{n+nb}{int}\PY{p}{(}\PY{n+nb}{input}\PY{p}{(}\PY{l+s+s1}{\PYZsq{}}\PY{l+s+s1}{Choose your next position: (1\PYZhy{}9) }\PY{l+s+s1}{\PYZsq{}}\PY{p}{)}\PY{p}{)}
                 
             \PY{k}{return} \PY{n}{position}
\end{Verbatim}


    \textbf{Step 9: Write a function that asks the player if they want to
play again and returns a boolean True if they do want to play again.}

    \begin{Verbatim}[commandchars=\\\{\}]
{\color{incolor}In [{\color{incolor}13}]:} \PY{k}{def} \PY{n+nf}{replay}\PY{p}{(}\PY{p}{)}\PY{p}{:}
             
             \PY{k}{return} \PY{n+nb}{input}\PY{p}{(}\PY{l+s+s1}{\PYZsq{}}\PY{l+s+s1}{Do you want to play again? Enter Yes or No: }\PY{l+s+s1}{\PYZsq{}}\PY{p}{)}\PY{o}{.}\PY{n}{lower}\PY{p}{(}\PY{p}{)}\PY{o}{.}\PY{n}{startswith}\PY{p}{(}\PY{l+s+s1}{\PYZsq{}}\PY{l+s+s1}{y}\PY{l+s+s1}{\PYZsq{}}\PY{p}{)}
\end{Verbatim}


    \textbf{Step 10: Here comes the hard part! Use while loops and the
functions you've made to run the game!}

    \begin{Verbatim}[commandchars=\\\{\}]
{\color{incolor}In [{\color{incolor}14}]:} \PY{n+nb}{print}\PY{p}{(}\PY{l+s+s1}{\PYZsq{}}\PY{l+s+s1}{Welcome to Tic Tac Toe!}\PY{l+s+s1}{\PYZsq{}}\PY{p}{)}
         
         \PY{k}{while} \PY{k+kc}{True}\PY{p}{:}
             \PY{c+c1}{\PYZsh{} Reset the board}
             \PY{n}{theBoard} \PY{o}{=} \PY{p}{[}\PY{l+s+s1}{\PYZsq{}}\PY{l+s+s1}{ }\PY{l+s+s1}{\PYZsq{}}\PY{p}{]} \PY{o}{*} \PY{l+m+mi}{10}
             \PY{n}{player1\PYZus{}marker}\PY{p}{,} \PY{n}{player2\PYZus{}marker} \PY{o}{=} \PY{n}{player\PYZus{}input}\PY{p}{(}\PY{p}{)}
             \PY{n}{turn} \PY{o}{=} \PY{n}{choose\PYZus{}first}\PY{p}{(}\PY{p}{)}
             \PY{n+nb}{print}\PY{p}{(}\PY{n}{turn} \PY{o}{+} \PY{l+s+s1}{\PYZsq{}}\PY{l+s+s1}{ will go first.}\PY{l+s+s1}{\PYZsq{}}\PY{p}{)}
             
             \PY{n}{play\PYZus{}game} \PY{o}{=} \PY{n+nb}{input}\PY{p}{(}\PY{l+s+s1}{\PYZsq{}}\PY{l+s+s1}{Are you ready to play? Enter Yes or No.}\PY{l+s+s1}{\PYZsq{}}\PY{p}{)}
             
             \PY{k}{if} \PY{n}{play\PYZus{}game}\PY{o}{.}\PY{n}{lower}\PY{p}{(}\PY{p}{)}\PY{p}{[}\PY{l+m+mi}{0}\PY{p}{]} \PY{o}{==} \PY{l+s+s1}{\PYZsq{}}\PY{l+s+s1}{y}\PY{l+s+s1}{\PYZsq{}}\PY{p}{:}
                 \PY{n}{game\PYZus{}on} \PY{o}{=} \PY{k+kc}{True}
             \PY{k}{else}\PY{p}{:}
                 \PY{n}{game\PYZus{}on} \PY{o}{=} \PY{k+kc}{False}
         
             \PY{k}{while} \PY{n}{game\PYZus{}on}\PY{p}{:}
                 \PY{k}{if} \PY{n}{turn} \PY{o}{==} \PY{l+s+s1}{\PYZsq{}}\PY{l+s+s1}{Player 1}\PY{l+s+s1}{\PYZsq{}}\PY{p}{:}
                     \PY{c+c1}{\PYZsh{} Player1\PYZsq{}s turn.}
                     
                     \PY{n}{display\PYZus{}board}\PY{p}{(}\PY{n}{theBoard}\PY{p}{)}
                     \PY{n}{position} \PY{o}{=} \PY{n}{player\PYZus{}choice}\PY{p}{(}\PY{n}{theBoard}\PY{p}{)}
                     \PY{n}{place\PYZus{}marker}\PY{p}{(}\PY{n}{theBoard}\PY{p}{,} \PY{n}{player1\PYZus{}marker}\PY{p}{,} \PY{n}{position}\PY{p}{)}
         
                     \PY{k}{if} \PY{n}{win\PYZus{}check}\PY{p}{(}\PY{n}{theBoard}\PY{p}{,} \PY{n}{player1\PYZus{}marker}\PY{p}{)}\PY{p}{:}
                         \PY{n}{display\PYZus{}board}\PY{p}{(}\PY{n}{theBoard}\PY{p}{)}
                         \PY{n+nb}{print}\PY{p}{(}\PY{l+s+s1}{\PYZsq{}}\PY{l+s+s1}{Congratulations! You have won the game!}\PY{l+s+s1}{\PYZsq{}}\PY{p}{)}
                         \PY{n}{game\PYZus{}on} \PY{o}{=} \PY{k+kc}{False}
                     \PY{k}{else}\PY{p}{:}
                         \PY{k}{if} \PY{n}{full\PYZus{}board\PYZus{}check}\PY{p}{(}\PY{n}{theBoard}\PY{p}{)}\PY{p}{:}
                             \PY{n}{display\PYZus{}board}\PY{p}{(}\PY{n}{theBoard}\PY{p}{)}
                             \PY{n+nb}{print}\PY{p}{(}\PY{l+s+s1}{\PYZsq{}}\PY{l+s+s1}{The game is a draw!}\PY{l+s+s1}{\PYZsq{}}\PY{p}{)}
                             \PY{k}{break}
                         \PY{k}{else}\PY{p}{:}
                             \PY{n}{turn} \PY{o}{=} \PY{l+s+s1}{\PYZsq{}}\PY{l+s+s1}{Player 2}\PY{l+s+s1}{\PYZsq{}}
         
                 \PY{k}{else}\PY{p}{:}
                     \PY{c+c1}{\PYZsh{} Player2\PYZsq{}s turn.}
                     
                     \PY{n}{display\PYZus{}board}\PY{p}{(}\PY{n}{theBoard}\PY{p}{)}
                     \PY{n}{position} \PY{o}{=} \PY{n}{player\PYZus{}choice}\PY{p}{(}\PY{n}{theBoard}\PY{p}{)}
                     \PY{n}{place\PYZus{}marker}\PY{p}{(}\PY{n}{theBoard}\PY{p}{,} \PY{n}{player2\PYZus{}marker}\PY{p}{,} \PY{n}{position}\PY{p}{)}
         
                     \PY{k}{if} \PY{n}{win\PYZus{}check}\PY{p}{(}\PY{n}{theBoard}\PY{p}{,} \PY{n}{player2\PYZus{}marker}\PY{p}{)}\PY{p}{:}
                         \PY{n}{display\PYZus{}board}\PY{p}{(}\PY{n}{theBoard}\PY{p}{)}
                         \PY{n+nb}{print}\PY{p}{(}\PY{l+s+s1}{\PYZsq{}}\PY{l+s+s1}{Player 2 has won!}\PY{l+s+s1}{\PYZsq{}}\PY{p}{)}
                         \PY{n}{game\PYZus{}on} \PY{o}{=} \PY{k+kc}{False}
                     \PY{k}{else}\PY{p}{:}
                         \PY{k}{if} \PY{n}{full\PYZus{}board\PYZus{}check}\PY{p}{(}\PY{n}{theBoard}\PY{p}{)}\PY{p}{:}
                             \PY{n}{display\PYZus{}board}\PY{p}{(}\PY{n}{theBoard}\PY{p}{)}
                             \PY{n+nb}{print}\PY{p}{(}\PY{l+s+s1}{\PYZsq{}}\PY{l+s+s1}{The game is a draw!}\PY{l+s+s1}{\PYZsq{}}\PY{p}{)}
                             \PY{k}{break}
                         \PY{k}{else}\PY{p}{:}
                             \PY{n}{turn} \PY{o}{=} \PY{l+s+s1}{\PYZsq{}}\PY{l+s+s1}{Player 1}\PY{l+s+s1}{\PYZsq{}}
         
             \PY{k}{if} \PY{o+ow}{not} \PY{n}{replay}\PY{p}{(}\PY{p}{)}\PY{p}{:}
                 \PY{k}{break}
\end{Verbatim}


    \begin{Verbatim}[commandchars=\\\{\}]
   |   |
   | O | O
   |   |
-----------
   |   |
   |   |  
   |   |
-----------
   |   |
 X | X | X
   |   |
Congratulations! You have won the game!
Do you want to play again? Enter Yes or No: No

    \end{Verbatim}

    \subsection{Good Job!}\label{good-job}


    % Add a bibliography block to the postdoc
    
    
    
    \end{document}
